\documentclass[11pt, oneside]{article}   	% use "amsart" instead of "article" for AMSLaTeX format
\usepackage{geometry}                		% See geometry.pdf to learn the layout options. There are lots.
\geometry{a4paper}                   		% ... or a4paper or a5paper or ...letterpaper 
%\geometry{landscape}                		% Activate for rotated page geometry
%\usepackage[parfill]{parskip}    		% Activate to begin paragraphs with an empty line rather than an indent
\usepackage{graphicx}				% Use pdf, png, jpg, or eps§ with pdflatex; use eps in DVI mode
								% TeX will automatically convert eps --> pdf in pdflatex		
\usepackage{amssymb}
\DeclareGraphicsExtensions{.pdf,.png,.jpg}
\graphicspath{ {images/} }

\usepackage{hyperref}
\usepackage{subcaption}


\title{Connecting Nutrition, Health and Environment }
\author{Perla Troncoso Rey, Wiktor Jurkowski, Earlham Institute}
\date{26 - 27 January 2017}	 % Activate to display a given date or no date

\begin{document}
\maketitle

\tableofcontents

\listoffigures
\listoftables


\part{Part I}

\section{Introduction to the data}

Some details

\subsection{Exploring the expression data}

\subsubsection{Principal Component Analysis}

\subsubsection{Hierarchical Clustering}

\subsection{Interaction Network}
\subsubsection{Compile the network with Cytoscape}


\part{Part II}
\section{Ranking genes (Feature Selection)} 

Feature selection with MultiPEN from expression leves and network 


\section{Pathway Analysis}

Using the ranking from feature selection


%Example to create a list
%List:
%\begin{itemize}
%	\item item 1
%	\item item 2		
%\end{itemize}


%Example to enumerate a list
%\begin{enumerate}	
%	\item enumerated item
%	\item another item
%\end{enumerate}


%Figure resference \autoref{fig:volcanoplot}). 
%
%\begin{figure}[!h]
%	\centering
%	\includegraphics[width=\textwidth]{VolcanoMatrix}
%	\caption{Matrix of volcano plots explore genes that differ significantly between pairs of conditions}
%	\label{fig:volcanoplot}
%\end{figure}

%figure with subfigures
%\begin{figure}[!h]
%	\centering
%	\begin{subfigure}{.3\textwidth}
%		\includegraphics[width=\textwidth]{MAPlotEPS1_8_EPS2_8}
%		\caption{MA for EPS1 and EPS2, 8 hours}
%		\label{fig:MAPlotEPS1_8_EPS2_8}
%	\end{subfigure}
%	\begin{subfigure}{0.3\textwidth}
%		\includegraphics[width=\textwidth]{MAPlotEPS1_8_noEPS_8}
%		\caption{MA for EPS1 and noEPS, 8 hours}
%		\label{fig:MAPlotEPS1_8_noEPS_8}
%	\end{subfigure}
%	\begin{subfigure}{.3\textwidth}
%		\includegraphics[width=\textwidth]{MAPlotEPS2_8_noEPS_8}
%		\caption{MA for EPS2 and noEPS, 8 hours}
%		\label{fig:MAPlotEPS2_8_noEPS_8}
%	\end{subfigure}
%	\caption{Average intensity vs log ratio Plots, for 8 hours}
%	\label{fig:MAPlot_8hours}
%\end{figure}


%See below example of table  \autoref{tab:table1}
%\begin{table}[h]
%	\centering
%	\caption{Example of table 1}
%	\begin{tabular}{c | c | c}
%		\hline
%		column 1 & column 2 & column 3 \\
%		\hline
%		aaaaa  & bbbbbbb & ccccccc \\ 
%		xxxxxxx  & yyyyyyyy & zzzzzzzz \\		
%		\label{tab:table1}
%	\end{tabular}
%\end{table}





\bibliography{references}
\bibliographystyle{plain}

\end{document}  